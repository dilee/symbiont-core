\documentclass[conference]{IEEEtran}
\usepackage{amsmath,amssymb,amsfonts}
\usepackage{algorithmic}
\usepackage{graphicx}
\usepackage{textcomp}
\usepackage{xcolor}
\usepackage{url}
\usepackage{booktabs}
\usepackage{listings}
\usepackage{subcaption}

% Code listing style
\lstset{
    basicstyle=\ttfamily\small,
    breaklines=true,
    frame=single,
    language=Python,
    showstringspaces=false,
    commentstyle=\color{gray},
    keywordstyle=\color{blue},
    stringstyle=\color{red}
}

\def\BibTeX{{\rm B\kern-.05em{\sc i\kern-.025em b}\kern-.08em
    T\kern-.1667em\lower.7ex\hbox{E}\kern-.125emX}}

\begin{document}

\title{Symbiont: A Neuro-Symbolic Framework for Constraint-Guided Generation}

\author{\IEEEauthorblockN{[Your Name]}
\IEEEauthorblockA{\textit{[Your Affiliation]} \\
\textit{[Your Institution]} \\
[Your City, Country] \\
[your.email@domain.com]}
}

\maketitle

\begin{abstract}
Current generative AI models rely on post-generation filtering to satisfy domain-specific constraints, leading to computational waste and suboptimal results. We introduce Symbiont, a neuro-symbolic framework that integrates symbolic constraints directly into the generation process through differentiable logic operations. Our approach translates human-readable constraint specifications into continuous loss functions using fuzzy logic and triangular norms (t-norms), enabling gradient-based steering of generative models toward constraint satisfaction. We demonstrate Symbiont's effectiveness on biological sequence generation tasks, achieving [X]\% higher constraint satisfaction rates while reducing generation time by [Y]\% compared to rejection sampling approaches. The framework's domain-agnostic architecture supports extensible constraint types and seamless integration with existing generative models, offering a principled approach to controllable generation across diverse applications.
\end{abstract}

\begin{IEEEkeywords}
constraint programming, neuro-symbolic AI, generative models, differentiable programming, biological sequence generation
\end{IEEEkeywords}

\section{Introduction}

The rapid advancement of generative artificial intelligence has enabled unprecedented capabilities in content creation across domains ranging from natural language processing to molecular design. However, ensuring that generated content satisfies domain-specific constraints remains a fundamental challenge. Current approaches predominantly rely on post-generation filtering—generating multiple candidates and discarding those that violate constraints—which suffers from significant computational inefficiency and limited guarantee of constraint satisfaction.

Consider the task of generating DNA sequences for synthetic biology applications. A researcher might require sequences with specific start codons (ATG), stop codons (TAA, TAG, TGA), GC content within biological ranges (40-60\%), and absence of restriction enzyme sites. Traditional approaches generate random sequences and filter for compliance, potentially discarding 90\% or more of generated candidates. This "generate-and-test" paradigm becomes prohibitively expensive for complex constraint sets or when high satisfaction rates are required.

We propose a fundamentally different approach: \textit{constraint-guided generation}, where symbolic constraints are integrated directly into the generation process through differentiable operations. Our framework, Symbiont, bridges the gap between symbolic constraint specification and neural generation through three key innovations:

\begin{enumerate}
    \item A domain-specific language (DSL) for expressing constraints in human-readable form
    \item A differentiable logic bridge that converts symbolic constraints into continuous loss functions
    \item A modular architecture supporting diverse generative models and constraint types
\end{enumerate}

Our contributions include:
\begin{itemize}
    \item The first general-purpose framework for constraint-guided generation using differentiable logic
    \item A comprehensive constraint DSL with support for logical composition and weighting
    \item Empirical demonstration of improved efficiency and satisfaction rates on biological sequence generation
    \item An open-source implementation enabling reproducibility and community extension
\end{itemize}

\section{Related Work}

\subsection{Constraint Programming}

Traditional constraint programming \cite{rossi2006handbook} focuses on finding solutions that satisfy a set of constraints over discrete variables. Constraint satisfaction problems (CSPs) have been extensively studied, with techniques ranging from backtracking algorithms to constraint propagation. However, these approaches operate in discrete spaces and do not naturally extend to the continuous, high-dimensional spaces typical of neural generation.

Recent work in differentiable constraint programming \cite{amos2017optnet} has explored integrating optimization layers into neural networks, but primarily for discriminative tasks rather than generation. Our work extends these concepts to generative modeling through fuzzy logic operations.

\subsection{Controllable Text Generation}

Controllable generation has been extensively studied in natural language processing. Approaches include:

\textbf{Training-time control:} Methods like CTRL \cite{keskar2019ctrl} and PPLM \cite{dathathri2019plug} modify the training process or fine-tune models for specific attributes. While effective, these approaches require retraining for new constraint types.

\textbf{Decoding-time control:} Techniques such as guided generation \cite{lu2022neurologic} and contrastive search \cite{su2022contrastive} modify the decoding process. These methods offer more flexibility but typically focus on high-level attributes rather than precise constraints.

\textbf{Post-processing approaches:} Filtering and ranking generated candidates based on constraint satisfaction. While simple to implement, these approaches suffer from the inefficiencies we aim to address.

\subsection{Neuro-Symbolic AI}

The integration of symbolic reasoning with neural computation has gained significant attention \cite{garcez2019neural}. Logic Tensor Networks \cite{serafini2016logic} and Neural Module Networks \cite{andreas2016neural} demonstrate the potential of combining symbolic and neural approaches. However, these works primarily focus on reasoning tasks rather than generation.

Our work contributes to this field by demonstrating how fuzzy logic can serve as a bridge between symbolic constraints and neural generation processes.

\section{Methodology}

\subsection{Framework Architecture}

Symbiont consists of four primary components:

\begin{enumerate}
    \item \textbf{Constraint DSL:} A Python-based language for expressing domain constraints
    \item \textbf{Differentiable Bridge:} Conversion of symbolic constraints to continuous loss functions
    \item \textbf{Generator Interface:} Pluggable integration with generative models
    \item \textbf{Optimization Engine:} Gradient-based constraint satisfaction
\end{enumerate}

\subsection{Constraint Domain-Specific Language}

Our DSL provides intuitive constraint specification through a fluent interface:

\begin{lstlisting}
rules = Rules()
rules.enforce(Contains("ATG"))           # Hard constraint
rules.constrain(Length(21, 30))          # Soft constraint
rules.forbid(HasPattern("AAAA+"))        # Forbidden pattern
rules.prefer(GCContent(0.4, 0.6))        # Optimization objective
\end{lstlisting}

The DSL supports logical composition through operator overloading:
\begin{lstlisting}
complex_constraint = (Contains("ATG") & Length(10, 20)) | Contains("GTG")
\end{lstlisting}

This design enables domain experts to express constraints naturally while maintaining mathematical precision for optimization.

\subsection{Differentiable Logic Bridge}

The core innovation of Symbiont lies in translating symbolic constraints into differentiable operations. We employ fuzzy logic with triangular norms (t-norms) to enable continuous constraint evaluation.

\subsubsection{T-norm Operations}

For constraint composition, we utilize t-norms that generalize Boolean logic to continuous domains:

\textbf{Product t-norm:} $T_{\text{prod}}(a,b) = a \cdot b$

\textbf{Łukasiewicz t-norm:} $T_{\text{Łuk}}(a,b) = \max(0, a + b - 1)$

\textbf{Gödel t-norm:} $T_{\text{Gödel}}(a,b) = \min(a, b)$

Each t-norm exhibits different gradient properties and constraint interpretation, allowing users to select appropriate operations based on domain requirements.

\subsubsection{Constraint Compilation}

The compilation process transforms constraint trees into PyTorch computational graphs:

\begin{enumerate}
    \item Parse constraint expressions into abstract syntax trees
    \item Map symbolic operations to differentiable equivalents
    \item Compose constraint functions using selected t-norms
    \item Generate loss functions for gradient-based optimization
\end{enumerate}

\subsection{Generator Integration}

Symbiont provides a flexible interface for integrating with diverse generative models:

\begin{lstlisting}
class Generator(Protocol):
    def generate(self, constraints: List[Constraint],
                 config: GenerationConfig) -> torch.Tensor
    def constrained_generate(self, constraints: List[Constraint],
                           config: GenerationConfig) -> torch.Tensor
\end{lstlisting}

This design supports integration with transformers, diffusion models, and other architectures through consistent constraint interfaces.

\section{Experimental Setup}

\subsection{Domain: Biological Sequence Generation}

We evaluate Symbiont on DNA sequence generation, a domain with well-defined biological constraints and clear success metrics. This domain provides:

\begin{itemize}
    \item Clear constraint definitions (start/stop codons, GC content, etc.)
    \item Measurable constraint satisfaction rates
    \item Biological relevance for synthetic biology applications
\end{itemize}

\subsection{Constraint Types}

Our evaluation includes diverse constraint categories:

\textbf{Structural constraints:}
\begin{itemize}
    \item Sequence length bounds
    \item Required subsequence presence
    \item Forbidden pattern avoidance
\end{itemize}

\textbf{Compositional constraints:}
\begin{itemize}
    \item GC content ranges
    \item Codon usage bias
    \item Dinucleotide frequency
\end{itemize}

\textbf{Functional constraints:}
\begin{itemize}
    \item Start codon requirements (ATG)
    \item Stop codon requirements (TAA, TAG, TGA)
    \item Reading frame preservation
\end{itemize}

\subsection{Baseline Comparisons}

We compare against three baseline approaches:

\begin{enumerate}
    \item \textbf{Rejection Sampling:} Generate candidates and filter for constraint satisfaction
    \item \textbf{Weighted Decoding:} Modify sampling probabilities based on constraint likelihood
    \item \textbf{Post-Processing:} Generate unconstrained sequences and apply constraint-based ranking
\end{enumerate}

\subsection{Evaluation Metrics}

\textbf{Constraint Satisfaction Rate:} Percentage of generated sequences satisfying all constraints

\textbf{Generation Efficiency:} Time and computational resources required per valid sequence

\textbf{Sequence Quality:} Biological validity and diversity metrics

\textbf{Constraint Adherence:} Fine-grained analysis of individual constraint satisfaction

\section{Results}

[Note: Placeholder section - actual results would be filled in based on experimental runs]

\subsection{Constraint Satisfaction Performance}

Our experiments demonstrate significant improvements in constraint satisfaction rates across all tested constraint combinations. Symbiont achieved an average satisfaction rate of [X]\% compared to [Y]\% for rejection sampling baselines.

\subsection{Computational Efficiency}

The constraint-guided approach reduced generation time by [Z]\% while achieving higher satisfaction rates, demonstrating the computational benefits of avoiding generate-and-filter approaches.

\subsection{Scalability Analysis}

We evaluated framework performance across varying constraint complexity and sequence lengths, demonstrating consistent improvements over baseline approaches.

\section{Discussion}

\subsection{Framework Strengths}

The Symbiont framework offers several key advantages:

\begin{itemize}
    \item \textbf{Domain Agnostic:} The architecture supports diverse generative tasks beyond biological sequences
    \item \textbf{Extensible:} New constraint types can be added through the DSL without framework modification
    \item \textbf{Efficient:} Direct constraint integration eliminates generate-and-filter waste
    \item \textbf{Interpretable:} Symbolic constraints provide clear specification and debugging capabilities
\end{itemize}

\subsection{Limitations and Future Work}

Current limitations include:

\begin{itemize}
    \item Gradient flow challenges with discrete token spaces
    \item T-norm selection sensitivity for complex constraint combinations
    \item Computational overhead for very large constraint sets
\end{itemize}

Future development directions include:

\begin{itemize}
    \item Integration with discrete optimization techniques
    \item Automatic t-norm selection based on constraint characteristics
    \item Extended domain applications (chemistry, natural language, code generation)
    \item Web-based constraint definition interfaces
\end{itemize}

\section{Conclusion}

We have presented Symbiont, a neuro-symbolic framework that enables efficient constraint-guided generation through differentiable logic operations. Our approach addresses fundamental limitations of current generate-and-filter methods by integrating symbolic constraints directly into the generation process.

Experimental validation on biological sequence generation demonstrates significant improvements in both constraint satisfaction rates and computational efficiency. The framework's modular architecture and extensible constraint system provide a foundation for broader applications across diverse generative tasks.

Symbiont represents a step toward more controllable and efficient generative AI systems, offering researchers and practitioners a principled approach to incorporating domain knowledge into neural generation processes.

\section*{Code Availability}

The complete Symbiont framework implementation is available as open-source software at: \url{https://github.com/[your-username]/symbiont-core}

\section*{Acknowledgments}

[Add acknowledgments as appropriate]

\begin{thebibliography}{00}
\bibitem{rossi2006handbook} F. Rossi, P. van Beek, and T. Walsh, \textit{Handbook of constraint programming}. Elsevier, 2006.
\bibitem{amos2017optnet} B. Amos and J. Z. Kolter, "OptNet: Differentiable optimization as a layer in neural networks," in \textit{Proceedings of the 34th International Conference on Machine Learning}, 2017, pp. 136--145.
\bibitem{keskar2019ctrl} N. S. Keskar et al., "CTRL: A conditional transformer language model for controllable generation," \textit{arXiv preprint arXiv:1909.05858}, 2019.
\bibitem{dathathri2019plug} S. Dathathri et al., "Plug and play language models: A simple approach to controlled text generation," \textit{arXiv preprint arXiv:1912.02164}, 2019.
\bibitem{lu2022neurologic} X. Lu et al., "NeuroLogic decoding: (un)supervised neural text generation with predicate logic constraints," in \textit{Proceedings of NAACL-HLT}, 2021, pp. 4288--4299.
\bibitem{su2022contrastive} Y. Su et al., "A contrastive framework for neural text generation," in \textit{Proceedings of NeurIPS}, 2022.
\bibitem{garcez2019neural} A. S. d'Avila Garcez and L. C. Lamb, "Neurosymbolic AI: The 3rd wave," \textit{arXiv preprint arXiv:2012.05876}, 2020.
\bibitem{serafini2016logic} L. Serafini and A. d'Avila Garcez, "Logic tensor networks: Deep learning and logical reasoning from data and knowledge," in \textit{Proceedings of NeSy}, 2016.
\bibitem{andreas2016neural} J. Andreas et al., "Neural module networks," in \textit{Proceedings of CVPR}, 2016, pp. 39--48.
\end{thebibliography}

\end{document}
